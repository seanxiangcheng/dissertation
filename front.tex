%% ------------------------ Distribution agreement page --------------------- %%
\makedistribution

%% -------------------------------- Title page ------------------------------ %%
\makeapproval

%% -------------------------------- Abstract cover page --------------------- %%
\makecover

%% ------------------------------ Abstract ---------------------------------- %%
\begin{abstract}
\noindent Physics is to study matters of the universe where 
there are tremendous more disordered materials than ordered ones.
For ordered physical systems, there has been a rich spectrum of 
well-defined theories, models, phases, and methods; while disordered 
systems has more questions that are unclear. In this work, we study 3 disordered systems in finite dimensional lattice-like structures, which may contribute new insights comparing to a large amount of work in mean-field-like models.

\vspace{5mm}
First, we apply a lattice glass model proposed by Biroli and Mezard onto
a number of hierarchical networks. These networks combine 
certain lattice-like features with a recursive structure that makes them 
suitable for exact renormalization group studies and provide 
an alternative to the mean-field approach. We explored both the equilibrium and dynamic behaviors and discover jamming transitions and no phase transitions. This discovery is the first evidence of a jamming transition with no phase transition.

\vspace{5mm}
Secondly, the antiferromagnetic Ising model (AFM) is a convenient model to introduce disorderedness and glassy dynamics.  We study the properties of the Ising antiferromagnet on 4 hierarchical networks that exhibit using both Monte Carlo methods and renormalization groups. 
Exact renormalization group calculations show that the system encounters an infinite-order
transition into a glassy state, characterized by a super-critical Hopf-bifurcation
in coupling-space to chaotic behavior for low temperatures.

\vspace{5mm}
Thirdly, random field Ising model (RFIM) is studied to understand the aging in an experimental system, a thin-film ferromagnet/antiferromagnet (F/AF) bilayer. The experiments show extremely slow cooperative relaxation. In our computational study, the experimental system is coarse-grained into a RFIM on a 2D square lattice. Monte Carlo simulations indicate that the aging process may be associated with the glassy evolution of the magnetic domain walls, due to the pinning by the random fields. The scaling of the simulated aging agrees well with experiments. Both are consistent with either a small power-law or logarithmic dependence on time.

\end{abstract}

%% -------------------------------- Abstract cover page --------------------- %%
\makecovertwo

%% ----------------------------- Acknowledgements --------------------------- %%
\begin{acknowledgements}
First of all, I would like to thank my advisor Dr. Stefan Boettcher first. He is a role model not only in research but also life attitudes. He always treats his students nicely and equally, like his friends. He always supports and encourages me to pursue my interests and to make reasonable plans. That is also an important reason why I received three job offers.  

Dr. Ilya Nemenman was my first research mentor in Emory. He has so many interesting ideas and alway inspired me to think deeply and critically. He has a high standard of research but also very helpful and supportive.

I would like to thank my other dissertation committee members. Dr. Fereydoon Family is so knowledgeable and caring and always ready to offer help. Dr. James Nagy brought me into math and computer science through his classes of Numerical Analysis. Dr. Nagy is also very helpful in helping me applying for summer fellowships and jobs.  Dr. Justin Burton is an expert in the disordered systems and always gave very good suggestions in the meetings with him.  

I would also like to thank all other faculty, staff, postdocs, and fellow students in Emory Physics, especially Dr. Laura Finzi, Dr. Sergei Urazhdin, Dr. Connie Roth, Dr. Kurt Warncke, Dr. Tom Bing, Dr. Eric Weeks, Cory Donofrio, Jason Boss, Art Kleyman, Calvin Jackson, Barbara Conner, Dr. Stefan Falkner, Dr. Martin Tchernookov, Dr. Lina Merchan, Dr. Trent Bruson, Dr. Xia Hong, Skanda Vivek, Shanshan Li, Nick Rob, Dr. Pascal Philipp, Baohua Zhou, Shengming Zhang, Yan Yan, KaWei Leung, Xinru Huang, Baohua Zhou, Dr. Xinxian Shao, Andrei Zholud, Roman Bagley, John Kirkham, and all others who helped me. 

Lastly, I want to thank my family. I am a first generation college student, and my parents have been so supportive and spent most savings on my education. My wife Xu Ji is a kindhearted, hard-working, and beautiful lady pursuing a PhD in Health Economics. After being in a relationship with her, I have been working much harder in research and career preparation, and became more organized in life. She is definitely a life partner who can make me a better person everyday.



\end{acknowledgements}
\thesistableofcontents
\thesislistoftables
\thesislistoffigures

%% ------------------------ Citation to previous work --------------------- %%
\begin{prev_citation}
\noindent Chapter 2 contains research from one publication:
\begin{itemize}
\item \underline{Xiang Cheng}, Stefan Boettcher, Jamming in Hierarchical Networks, {\it Comput. Phys. Commun.} {\bf 196},19-26, (2015)
\end{itemize}

\noindent Research in Chapter 3 preparing for publication:
\begin{itemize}
\item Stefan Boettcher,  \underline{Xiang Cheng}, Antiferromagnetic Ising Model in Hanoi Networks, in preparation, (2016)
\end{itemize}

\noindent Chapter 4 contains research from one publication:
\begin{itemize}
\item  Tianyu Ma, \underline{Xiang Cheng}, Stefan Boettcher, Sergei Urazhdin, Thickness-dependent cooperative aging in polycrystalline films of antiferromagnet, {\it Phys. Rev. B},   {\bf 94}, 024422 (2016)
\end{itemize}


\noindent Published work not included in this dissertation:
\begin{itemize}
\item  \underline{Xiang Cheng}, Lina Merchan, Martin Tchernookov, Ilya Nemenman, A large number of receptors may reduce cellular response time variation, {\it Phys. Biol.}  {\bf 10},3 (2013)

\item  Michael Madaio, Shang-Tse Chen, Oliver L Haimson, Wenwen Zhang, \underline{Xiang Cheng}, Matthew Hinds-Aldrich, Duen Horng Chau, Bistra Dilkina, Firebird: Predicting Fire Risk and Prioritizing Fire Inspections in Atlanta, {\it  Knowledge Discovery and Data Mining 2016} (KDD 2016) \underline{\it Best Student Paper (runner-up)}
\end{itemize}

\end{prev_citation}

\pagestyle{fancy}
