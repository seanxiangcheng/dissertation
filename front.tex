%% ------------------------ Distribution agreement page --------------------- %%
\makedistribution

%% -------------------------------- Title page ------------------------------ %%
\makeapproval

%% -------------------------------- Abstract cover page --------------------- %%
\makecover

%% ------------------------------ Abstract ---------------------------------- %%
\begin{abstract}
\noindent Physics is to study matters of the universe where 
there are tremendous more disordered materials than ordered ones.
For ordered physical systems, there has been a rich specturm of 
well-defined theories, models, phases, and methods; while disordered 
systems has more questions that are unclear. In this work, we study 3 disordered systems in finite dimensional lattice-like structures, which may contribute new insights comparing to a large amount of work in mean-field-like models.

\vspace{5mm}
First, we apply a lattice glass model proposed by Biroli and Mezard onto
a number of hierarchical networks, called Hanoi Networks (HNs). These networks combine 
certain lattice-like features with a recursive structure that makes them 
suitable for exact renormalization group studies and provide 
an alternative to the mean-field approach. In our 
numerical simulations, we first explore their equilibrium 
properties with the Wang-Landau algorithm. Then, we investigate
their dynamical behavior using a grand-canonical annealing algorithm.
We find that the dynamics readily falls out of equilibrium and jams
in many of our networks with certain constraints on the neighborhood
occupation imposed by the Biroli-Mezard model, even in cases where
exact results indicate that no ideal glass transition exists. But
while we find that time-scales for the jams diverge, our simulations
can not ascertain such a divergence for a packing fraction  distinctly 
above random close packing. In cases where we allow hopping in our 
dynamical simulations, the jams on these networks generally disappear.

\vspace{5mm}
Secondly, the antiferromagnetic Ising model (AFM) is a convenient model to introduce disorderness and glassy dynamics.  The geometric frustrations in AMF may give rise to a spin glass phase and glassy relaxation at low temperatures.  We study the AFM in 4 HNs using both exact renormalization group analysis and numerical simulations. We first explore the dynamical behaviors using simulated annealing and discover an extremely slow relaxation at low temperatures. Then we employ the Wang-Landau algorithm to investigate the energy landscape and the corresponding equilibrium behaviors for different system sizes. Besides the Monte Carlo methods, renormalization group is used to study the equilibrium properties in the thermodynamic limit, and we find spin glass phases and an interesting phase diagram.

\vspace{5mm}
Thirdly, random field Ising model is another popular disordered system. The random fields introduced into the classical Ising models can roughen the energy landscape, leading to complex nonequilibrium dynamics. The effects of random fields on magnetism have been previously studied in the context of dilute antiferromagnets (AF), impure substrates, and magnetic alloys. We utilized random-field spin models to simulate the observed magnetic aging in thin-film ferromagnet/antiferromagnet (F/AF) bilayers. The experiments show extremely slow cooperative relaxation over a wide range of temperatures and magnetic fields. In our computational study, the experimental system is coarse-grained into a random field Ising model on a 2D square lattice. Monte Carlo simulations indicate that aging processes may be associated with the glassy evolution of the magnetic domain walls, due to the pinning by the random fields. The scaling of the simulated aging agrees well with experiments. Both are consistent with either a small power-law or logarithmic dependence on time.

\end{abstract}

%% -------------------------------- Abstract cover page --------------------- %%
\makecovertwo

%% ----------------------------- Acknowledgements --------------------------- %%
\begin{acknowledgements}
Thank you! To Be Completed! 
\end{acknowledgements}
\thesistableofcontents
\thesislistoftables
\thesislistoffigures

%% ------------------------ Citation to previous work --------------------- %%
\begin{prev_citation}
\noindent Chapter 2 contains research from one publication:
\begin{itemize}
\item \underline{Xiang Cheng}, Stefan Boettcher, Jamming in Hierarchical Networks, {\it Comput. Phys. Commun.} {\bf 196},19-26, (2015)
\end{itemize}

\noindent Research in Chapter 3 preparing for publication:
\begin{itemize}
\item \underline{Xiang Cheng}, Stefan Boettcher, Antiferromagnetic Ising Model in Hanoi Networks, draft preparing, (2016)
\end{itemize}

\noindent Chapter 4 contains research from one publication:
\begin{itemize}
\item  Tianyu Ma, \underline{Xiang Cheng}, Stefan Boettcher, Sergei Urazhdin, Thickness-dependent cooperative aging in polycrystalline films of antiferromagnet, {\it Phys. Rev. B},   {\bf 94}, 024422 (2016)
\end{itemize}


\noindent Published work not included in this dissertation:
\begin{itemize}
\item  \underline{Xiang Cheng}, Lina Merchan, Martin Tchernookov, Ilya Nemenman, A large number of receptors may reduce cellular response time variation, {\it Phys. Biol.}  {\bf 10},3 (2013)

\item  Michael Madaio, Shang-Tse Chen, Oliver L Haimson, Wenwen Zhang, \underline{Xiang Cheng}, Matthew Hinds-Aldrich, Duen Horng Chau, Bistra Dilkina, Firebird: Predicting Fire Risk and Prioritizing Fire Inspections in Atlanta, {\it  Knowledge Discovery and Data Mining 2016} (KDD 2016) \underline{\it Best Student Paper (runner-up)}
\end{itemize}

\end{prev_citation}

\pagestyle{fancy}
